%%%%%%%%%%%%%%%%%%%%%%%%%%%%%%%%%%%%%%%%%
% "ModernCV" CV and Cover Letter
% LaTeX Template
% Version 1.1 (9/12/12)
%
% This template has been downloaded from:
% http://www.LaTeXTemplates.com
%
% Original author:
% Xavier Danaux (xdanaux@gmail.com)
%
% License:
% CC BY-NC-SA 3.0 (http://creativecommons.org/licenses/by-nc-sa/3.0/)
%
% Important note:
% This template requires the moderncv.cls and .sty files to be in the same 
% directory as this .tex file. These files provide the resume style and themes 
% used for structuring the document.
%
%%%%%%%%%%%%%%%%%%%%%%%%%%%%%%%%%%%%%%%%%

%----------------------------------------------------------------------------------------
%	PACKAGES AND OTHER DOCUMENT CONFIGURATIONS
%----------------------------------------------------------------------------------------

\documentclass[12pt,a4paper,sans]{moderncv} % Font sizes: 10, 11, or 12; paper sizes: a4paper, letterpaper, a5paper, legalpaper, executivepaper or landscape; font families: sans or roman
\usepackage{standalone}
\moderncvstyle{classic} % CV theme - options include: 'casual' (default), 'classic', 'oldstyle' and 'banking'
\moderncvcolor{blue} % CV color - options include: 'blue' (default), 'orange', 'green', 'red', 'purple', 'grey' and 'black'

\usepackage{lipsum} % Used for inserting dummy 'Lorem ipsum' text into the template

\usepackage[scale=0.85]{geometry} % Reduce document margins
%\setlength{\hintscolumnwidth}{3cm} % Uncomment to change the width of the dates column
%\setlength{\makecvtitlenamewidth}{10cm} % For the 'classic' style, uncomment to adjust the width of the space allocated to your name

%\usepackage[utf8]{inputenc}

%\usepackage{booktabs}
\usepackage{fontawesome}
\usepackage{marvosym} % For cool symbols.
%\usepackage{hyperref}



%----------------------------------------------------------------------------------------
%	NAME AND CONTACT INFORMATION SECTION
%----------------------------------------------------------------------------------------

\firstname{Hanjing} % Your first name
\familyname{YE} % Your last name

% All information in this block is optional, comment out any lines you don't need
\title{Curriculum Vitae}
\address{Department of Electrical and Electronic Engineering}{Southern University of Science and Technology}
% \mobile{(+91) 0000000, (+91) 0000000}


%\fax{(000) 111 1113}
 
% \social{github}{stefano-bragaglia}
\email{yehj2022@mail.sustech.edu.cn} 



\homepage{medlartea.github.io}{My Webpage}

% \extrainfo{\faGithub\href{https://github.com/MedlarTea}{ Github}}
% social link \faGithub, \faSkype, \faLinkedin,\faStackExchange, and \faStackOverflow
% \extrainfo{
%     \faGithub\href{https://github.com/MedlarTea}{ Github} \quad
%     \ffagoogle\href{https://www.researchgate.net/profile/Hanjing-Ye}{ ResearchGate} \quad
%     \faSkype\href{https://skype.com/abc}{Skype}
%     }



%\social[linkedin][www.linkedin.com]{name}
% The first argument is %the url for the clickable link, the second argument is the url displayed in the %template - this allows special characters to be displayed such as the tilde in this %example

\photo[70pt][0.3pt]{picture} % The first bracket is the picture height, the second is %the thickness of the frame around the picture (0pt for no frame)
%\quote{Not Attention, Patience is all we need.}

%----------------------------------------------------------------------------------------

\newcommand{\cvdoublecolumn}[2]{%
  \cvitem[.75em]{}{%
    \begin{minipage}[t]{\listdoubleitemcolumnwidth}#1\end{minipage}%
    \hfill%
    \begin{minipage}[t]{\listdoubleitemcolumnwidth}#2\end{minipage}%
    }%
}



\usepackage{multibbl}
\newcommand\Colorhref[3][orange]{\href{#2}{\small\color{#1}#3}}


% \newcommand{\cvreference}[7]{%
%     \textbf{#1}\newline% Name
%     \ifthenelse{\equal{#2}{}}{}{\addresssymbol~#2\newline}%
%     \ifthenelse{\equal{#3}{}}{}{#3\newline}%
%     \ifthenelse{\equal{#4}{}}{}{#4\newline}%
%     \ifthenelse{\equal{#5}{}}{}{#5\newline}%
%     \ifthenelse{\equal{#6}{}}{}{\emailsymbol~\texttt{#6}\newline}%
%     \ifthenelse{\equal{#7}{}}{}{\phonesymbol~#7}}

\begin{document}

\makecvtitle % Print the CV title




%----------------------------------------------------------------------------------------
%	EDUCATION SECTION
%----------------------------------------------------------------------------------------

\section{Education}

\cventry{2022--present}{PhD candidate, Electrical and Electronic Engineering}{Southern University of Science and Technology (SUSTech)}{Shenzhen}{}
{}  % Arguments not required can be left empty

\cventry{2020--2022 :}{Visisting student, Electrical and Electronic Engineering}{SUSTech}{Shenzhen}{}
{}

\cventry{2019--2021 :}{Master of Engineering, Mechanical and Electrical Engineering}{Guangdong University of Technology (GDUT)}{Guangzhou}{}{}

\cventry{2015--2019 :}{Bachelor of Engineering, Mechanical and Electrical Engineering}{GDUT}{Guangzhou}{}{}
%{Advanced exposure to various areas of computer science along with a one and half year research project on Reversible Logic Synthesis.}
%\cvitem{CGPA :}{7.96/10}
% \cventry{2009--2013 :}{Bachelor of Engineering, Computer Science \& Technology}{Indian Institute of Engineering Science \& Technology}{Shibpur(\textit{Formerly} Bengal Engineering and Science University, Shibpur)}{}{}
%{Comprehensive exposure to the core areas of Computer Science along with a final year project on Data-mining}
%\cvitem{CGPA :}{7.36/10}
% \cventry{2008 :}{Higher Secondary Examination}{Belmuri Union Institution}{Belmuri}{}{ Mathematics, Physics, Chemistry, Biology, English, Bengali}
% {}
% \cvitem{Percentage :}{81.2 \%}
% \cventry{2006 :}{Secondary Examination}{Belmuri Union Institution}{Belmuri}{}{ Mathematics, Physical Science, Life Science, Geography, History, English, Bengali}
% {}
% \cvitem{Percentage :}{90.8 \%}





%----------------------------------------------------------------------------------------
%	PUBLICATION SECTION
%----------------------------------------------------------------------------------------


\section{Publications}
% \subsection{Journal Article(Accepted)}
% \cventry{2019}{\textbf{Pratik Dutta}, Sriparna Saha, Sanket Pai and Aviral Kumar}{}{Protein-protein Interaction based Generative Model for Improving Gene Clustering}{In \textit{\textbf{Scientific Reports-Nature}} (\textbf{Impact Factor: 4.12)}}{}
\cvitem{}{$^{\dagger}$ indicates equal contribution, and $^{*}$ indicates corresponding authorship.}


\subsection{Conference Proceedings}
\newbibliography{conference}
\nocite{conference}{*}
\bibliographystyle{conference}{plainyrrev}
\bibliography{conference}{conference}
{\large \textsc{Refereed Conference Publications}}
\cventry{2024}{\textmd{Jieting Zhao,} Hanjing Ye, \textmd{Yu Zhan and Hong Zhang$^{*}$}}{}{Human Orientation Estimation Under Partial Observation}{Submitted to \textit{2024 IEEE International Conference on Intelligent Robots and Systems (IROS)} -- Submitted}{}
\cventry{2024}{\textmd{Jingwen Yu,} Hanjing Ye, \textmd{Jianhao Jiao, Ping Tan and Hong Zhang$^{*}$}}{}{GV-Bench: Benchmarking Local Feature Matching for Geometric Verification of Long-term Loop Closure Detection}{Submitted to \textit{2024 IEEE International Conference on Intelligent Robots and Systems (IROS)} -- Submitted}{}
\cventry{2023}{Hanjing Ye, \textmd{Jieting Zhao, Yaling Pan, Weinan Chen, Li He and Hong Zhang$^{*}$}}{}{Robot Person Following Under Partial Occlusion}{In \textit{2023 IEEE International Conference on Robotics and Automation (ICRA)} -- Published}{}
\cventry{2023}{\textmd{Zhilong Tang}, Hanjing Ye \textmd{and Hong Zhang$^{*}$}}{}{Multi-scale Point Octree Encoding Network for Point Cloud based Place Recognition}{In \textit{2023 IEEE International Conference on Intelligent Robots and Systems (IROS)} -- Published}{}
\cventry{2022}{\textmd{Weinan Chen$^{\dagger}$,} Hanjing Ye$^{\dagger}$, \textmd{Lei Zhu, Chao Tang, Changfei Fu and Hong Zhang$^{*}$}}{}{Keyframe Selection with Information Occupancy Grid Model for Long-term Data Association}{In \textit{2022 IEEE International Conference on Intelligent Robots and Systems (IROS)} -- Published}{}
\cventry{2021}{Hanjing Ye, \textmd{Guangcheng Chen, Weinan Chen, Li He, Yisheng Guan and Hong Zhang$^{*}$}}{}{Mapping While Following: 2D LiDAR SLAM in Indoor Dynamic Environments with a Person Tracker}{In \textit{2021 IEEE International Conference on Robotics and Biomimetics (ROBIO)} -- Published}{}

\subsection{Journal Articles}
\newbibliography{journal}
\nocite{journal}{*}
\bibliographystyle{journal}{plainyrrev}
\bibliography{journal}{journal}
{\large \textsc{Refereed Journal Articles}}
\cventry{2024}{Hanjing Ye, \textmd{Jieting Zhao, Yu Zhan, Weinan Chen, Li He and Hong Zhang$^{*}$}}{}{Person Re-Identification for Robot Person Following with Online Continual Learning}{Submitted to \textit{IEEE Robotics and Automation Letters} -- Re-submitted}{}

\cventry{2023}{Hanjing Ye$^{\dagger}$, \textmd{Weinan Chen$^{\dagger}$, Jingwen Yu, Li He, Yisheng Guan and Hong Zhang$^{*}$}}{}{Condition-Invariant and Compact Visual Place Description by Convolutional Autoencoder}{\textit{ROBOTICA} -- Published}{}

















%----------------------------------------------------------------------------------------
%	WORK EXPERIENCE SECTION
%----------------------------------------------------------------------------------------

\section{Research Experience}
\subsection{Shenzhen Key Laboratory of Robotics and Computer Vision, SUSTech}
\cventry{2021--present}{\textit{Robot Person Following (RPF)}}{}{}{}
{
  \begin{itemize}
    \item Developing a novel method for natural person following, including the creation of a dataset and an imitation learning framework to enhance the generalization of social understanding.
    \item Developing an active person recovery method integrating motion cues and person verification uncertainty in an unknown environment.
    \item Proposed a robust person re-identification framework capable of adapting to severe domain drifts through online continual learning.
    \item Proposed a vision-based RPF system to locate and follow an user, effectively handling partial occlusions using a joint-height-based geometric model.
    \item Developed an RPF-assisted 2D LiDAR SLAM system, streamlining the mapping process and reducing the impact of dynamic objects.
  \end{itemize}
  % Developing a deep multi-modal architecture for accurately predicting protein interaction information from biomedical text. 
}
\cvitem{Advisor :}{\textbf{Dr. Hong Zhang}, \textit{Chair Professor, Department of Electronic and Electrical Engineering, SUSTech}; Professor Emeritus, University of Alberta}

\subsection{Shenzhen Key Laboratory of Robotics and Computer Vision, SUSTech \&  Biomimetic and Intelligent Robotics Laboratory, GDUT}

\cventry{2019--2021}{\textit{Visual Place Recognition (VPR)}}{}{}{}
{
  \begin{itemize}
    \item Introduced a keyframe selection strategy leveraging an information occupancy grid model. This approach, based on explainable deep learning descriptors and information gain theory, enhances long-term data association.
    \item Proposed a condition-invariant, compact visual place description for VPR, employing a convolutional-autoencoder-based reconstruction process to distill high-level representations.
  \end{itemize}
  % Analyzing different modalities of genes like gene expression profiles, protein 3D structure, underlying amino acid sequence using popular deep learning models to obtain deeper insight into the underlying biological system. 
}
\cvitem{Advisor :}{\textbf{Dr. Hong Zhang}, \textit{Chair Professor, Department of Electronic and Electrical Engineering, SUSTech}; Professor Emeritus, University of Alberta}




% \subsection{Indian Institute of Technology, XYZ}
% \cventry{January,2015 -- Dec,2015}{\textit{Design and Synthesis of Reversible Multi-dimentional Nearest-Neighbour(NN) Quantum Circuit}}{}{}{}{Proposed an approach for designing and physically implementing of the multi-dimensional quantum circuits maintaining nearest-neighbor complacency that use minimal number of SWAP gates.}
% \cvitem{Advisor :}{\textbf{Dr. abc xyz}, \textit{Associate Professor, Department of Computer Science \& Engineering}, IIT abc ({\Colorhref{https://www.personal_webpage.com/} {\textit{Personal Web-page}}})}


% \cventry{2012 -- 2013}{\textit{Text Document Clustering with Semantic Similarity through Wordnet}}{}{}{}{Improvement of the text document clustering task over conventional methods by introducing WORDNET and some better clustering algorithms.}
% \cvitem{Advisor :}{\textbf{Dr. abc xyz}, \textit{Associate Professor, Department of Computer Science \& Engineering}, IIT abc ({\Colorhref{https://www.personal_webpage.com/} {\textit{Personal Web-page}}})}











%----------------------------------------------------------------------------------------
%	Fellowships \& Awards
%----------------------------------------------------------------------------------------

% \section{Fellowships \& Awards}

% \cvitem{2016 --present}{\textit{\textbf{Visvesvaraya Fellowship}} of Ministry of Electronics and Information Technology (MeitY), Government of India, as a PhD research scholar in Indian Institute of Technology Patna.}
% \cvitem{2019}{Receipt of \textit{\textbf{Visvesvaraya Travel Grant}} to attend a international conference \textbf{\textit{IEEE Congress on Evolutionary Computation, 2019}} in Wellington, New Zealand.}
% \cvitem{2018}{Recipient of \textit{\textbf{SciGenome Research Foundation (SGRF) GYAN Scholarship}} to participate \textbf{\textit{Nextgen Genomics, Biology, Bioinformatics and Technologies-2018}} meeting at Jaipur India from $30^{th}$ September to $2^{nd}$ October 2018. }
% \cvitem{2015}{Awarded under \textit{\textbf{Students Reward Programme}} at the Annual General Meeting of \textbf{Global Alumni Association of Bengal Engineering and Science University(GAABESU).}}


%----------------------------------------------------------------------------------------
%	Academic achievements
%----------------------------------------------------------------------------------------

% \section{Academic Achievements \& Recognitions }


% \cvitem{2018}{\textbf{Session Chair of the session "Prediction"} in \textbf{$25^{th}$ \textit{International Conference of Neural Information Processing} (ICONIP 2018)}, Siem Reap, Cambodia.}


% \cvitem{2018}{Invited to conduct lab sessions in \textit{\textbf{"Training Program on Machine Learning For Ocean Acoustics and Climate Data Analysis"}}, during 22-36 October 2018 at \textbf{Defence R\&D Organization- Naval Physical \& Oceanographic Laboratory (DRDO-NPOL), Kochi, Kerala}.}




%----------------------------------------------------------------------------------------
%	COMPUTER SKILLS SECTION
%----------------------------------------------------------------------------------------

% \section{Computer skills}

% \cvitem{Programming Languages}{Python, PyTorch, keras, R, C, C++, Advanced JAVA}
% \cvitem{Web Technologies}{HTML 5, PHP, JSP, Javascript}
% \cvitem{Database}{SQL, MySQL, Apache, Neo4j}


%----------------------------------------------------------------------------------------
%	Position of Responsibility SECTION
%----------------------------------------------------------------------------------------

% \section{Position of Responsibility}
% \cventry{2016-2020}{Executive member of IEEE Student Branch}{}{IIT ABC}{}{}
% \cventry{April 1-5, 2019}{Organizer, GIAN Workshop on subjects}{}{IIT ABC}{}{}




%----------------------------------------------------------------------------------------
%	Teaching Assistantship SECTION
%----------------------------------------------------------------------------------------

\section{Teaching Assistant}
% \cventry{Fall, 2019 :}{CS564: Foundations of Machine Learning}{}{IIT ABC}{}{}
\cventry{Fall, 2022:}{EE5346: Autonomous Robot Navigation}{}{SUSTech}{}{}
\cventry{Spring, 2021:}{EE346: Mobile Robot Navigation and Control}{}{SUSTech}{}{}



% \section{Referees}


% \begin{tabular}{lr}
% % Referee 1
% \begin{minipage}[t]{3in}
% \textbf{Dr. XXXXX XXXXX}\\
% \textit{Associate Professor, Department of} \\
% \textit{Computer Science \& Engineering}\\
% Institute name\\
% \Letter\ \href{mailto:abc@gmail.com}{abc@gmail.com}
% \end{minipage}
% &
% % Referee 2
% \begin{minipage}[t]{3in}
% \textbf{Dr. XXXXX XXXXX}\\
% \textit{Associate Professor, Department of} \\
% \textit{Computer Science \& Engineering}\\
% Institute name\\
% \Telefon\ +(601) 877-6236\\
% \Letter\ \href{mailto:abc@gmail.com}{abc@gmail.com}
% \end{minipage}
% \\
% \\ % Additional newline for spacing.
% % Referee 3
% \begin{minipage}[t]{3in}
% \textbf{Dr. XXXXX XXXXX}\\
% \textit{Associate Professor, Department of} \\
% \textit{Computer Science \& Engineering}\\
% Institute name\\
% \Telefon\ +(601) 877-6236\\
% \Letter\ \href{mailto:abc@gmail.com}{abc@gmail.com}
% \end{minipage}
% &
% \\
% \end{tabular}


\end{document}